\documentclass[12pt]{article}
\usepackage[utf8]{inputenc}
\usepackage[brazil]{babel}
\usepackage{geometry}
\usepackage{array}
\usepackage{multirow} 
\usepackage{graphicx}
\usepackage{pdflscape}
\usepackage{booktabs}
\usepackage{multicol}
\usepackage{parskip}
\usepackage{varwidth}
\usepackage{tabularx}
\usepackage{helvet}
\usepackage[table]{xcolor}
\definecolor{roxo}{HTML}{7b4db7}
\renewcommand{\familydefault}{\sfdefault}
\geometry{a4paper, margin=1.5cm}
\pagestyle{empty}

\begin{document}
\begin{landscape}

% --- Cabeçalho ---
\begin{flushleft}
    \begin{tabular}{@{}l l@{}}
        \raisebox{0pt}[0pt][0pt]{\includegraphics[height=2cm]{escala/recursos/ufac-logo.jpeg}} &
        \parbox[b][2cm][t]{0.82\linewidth}{
            \raggedright
            {\fontsize{11pt}{13pt}\selectfont
                \textbf{RESIDÊNCIA EM ENFERMAGEM OBSTÉTRICA} \\
                \textbf{Escala de Trabalho e Atividades de Estudo} \\
                Mês:\textbf{ {{ mes }} / {{ ano }} }\\
                \textbf{Instituição:} {{ instituicao }}
            }
        }
    \end{tabular}

    {\fontsize{12pt}{14pt}\selectfont
        \textbf{Residente: {{ residente }}} \\
        Setores: {{ setores }}\\
        Preceptora: {{ preceptores }}
    }
\end{flushleft}

% --- Tabela de Escala ---
{\fontsize{8pt}{10pt}\selectfont
\begin{center}
\renewcommand{\arraystretch}{1.6}
\setlength{\tabcolsep}{3.7pt}

\begin{tabular}{|c|*{30}{>{\centering\arraybackslash}p{0.6cm}|}}
\hline
\multirow{2}{*}{\textbf{\shortstack{D\\i\\a}}} 
& \textbf{\cellcolor{roxo}1} & \textbf{2} & \textbf{3} & \textbf{4} & \textbf{5} & \textbf{6} & \textbf{\cellcolor{roxo}7} 
& \textbf{\cellcolor{roxo}8} & \textbf{9} & \textbf{10} & \textbf{11} & \textbf{12} & \textbf{13} & \textbf{\cellcolor{roxo}14} 
& \textbf{\cellcolor{roxo}15} & \textbf{16} & \textbf{17} & \textbf{18} & \textbf{19} & \textbf{20} & \textbf{\cellcolor{roxo}21} 
& \textbf{\cellcolor{roxo}22} & \textbf{23} & \textbf{24} & \textbf{25} & \textbf{26} & \textbf{27} & \textbf{\cellcolor{roxo}28}
& \textbf{\cellcolor{roxo}29} & \textbf{30} \\
\cline{2-31}
& \textbf{\cellcolor{roxo}D} & \textbf{S} & \textbf{T} & \textbf{Q} & \textbf{Q} & \textbf{S} & \textbf{\cellcolor{roxo}S}
& \textbf{\cellcolor{roxo}D} & \textbf{S} & \textbf{T} & \textbf{Q} & \textbf{Q} & \textbf{S} & \textbf{\cellcolor{roxo}S} 
& \textbf{\cellcolor{roxo}D} & \textbf{S} & \textbf{T} & \textbf{Q} & \textbf{Q} & \textbf{S} & \textbf{\cellcolor{roxo}S} 
& \textbf{\cellcolor{roxo}D} & \textbf{S} & \textbf{T} & \textbf{Q} & \textbf{Q} & \textbf{S}  & \textbf{\cellcolor{roxo}S} 
& \textbf{\cellcolor{roxo}D} & \textbf{S} \\
\hline
\textbf{M} 
& \textbf{\cellcolor{roxo}M} & \textbf{SM} & \textbf{SM} & \textbf{SM} & \textbf{APN} & \textbf{APN} & \textbf{\cellcolor{roxo}SM} 
& \textbf{\cellcolor{roxo}SM} & \textbf{SM} & \textbf{ } & \textbf{ } & \textbf{ } & \textbf{ } & \textbf{\cellcolor{roxo} }
& \textbf{\cellcolor{roxo} } & \textbf{ } & \textbf{ } & \textbf{ } & \textbf{ } & \textbf{ } & \textbf{\cellcolor{roxo} } 
& \textbf{\cellcolor{roxo} } & \textbf{ } & \textbf{ } & \textbf{ } & \textbf{ } & \textbf{ } & \textbf{\cellcolor{roxo} } 
& \textbf{\cellcolor{roxo} } & \textbf{ }\\
\hline
\textbf{T} 
& \textbf{\cellcolor{roxo}PPP} & \textbf{PPP} & \textbf{PPP} & \textbf{PPP} & \textbf{EAP} & \textbf{PPP} & \textbf{\cellcolor{roxo}AES} 
& \textbf{\cellcolor{roxo}MPI} & \textbf{APN} & \textbf{APN} & \textbf{PPP} & \textbf{PPP} & \textbf{MPI} & \textbf{\cellcolor{roxo}PPP} 
& \textbf{\cellcolor{roxo}PPP} & \textbf{PPP} & \textbf{PPP} & \textbf{MPI} & \textbf{ } & \textbf{ } &  \textbf{\cellcolor{roxo} } 
& \textbf{\cellcolor{roxo} } & \textbf{ } & \textbf{ } & \textbf{ } & \textbf{ } & \textbf{ } & \textbf{\cellcolor{roxo} } 
& \textbf{\cellcolor{roxo} } & \textbf{ }\\
\hline
\textbf{N} 
& \textbf{\cellcolor{roxo}CCH} & \textbf{CCH} & \textbf{CCH} & \textbf{?} & \textbf{CCH} & \textbf{CCH} & \textbf{\cellcolor{roxo}CCH} 
& \textbf{\cellcolor{roxo}CCH} & \textbf{ } & \textbf{ } & \textbf{ } & \textbf{ } & \textbf{ } & \textbf{\cellcolor{roxo} }
& \textbf{\cellcolor{roxo} } & \textbf{ } & \textbf{ } & \textbf{ } & \textbf{ } & \textbf{ } & \textbf{\cellcolor{roxo} } 
& \textbf{\cellcolor{roxo} } & \textbf{ } & \textbf{ } & \textbf{ } & \textbf{ } & \textbf{ } & \textbf{\cellcolor{roxo} } 
& \textbf{\cellcolor{roxo} } & \textbf{ }\\
\hline
\end{tabular}
\end{center}
}

\vspace{0.5cm}

% --- Memória de cálculo da carga horária ---
{\fontsize{10pt}{12pt}\selectfont
\noindent
\begin{tabularx}{\linewidth}{@{}X r@{}}
Memória de cálculo da carga horária hospital:
{{ memoria_descritivo_hospital }}&
Total de horas práticas: {{ memoria_total_horas_hospital }}h\\
\end{tabularx}
\noindent
\begin{tabularx}{\linewidth}{@{}X r@{}}
Memória de cálculo da carga horária teórica: 
{{ memoria_descritivo_teorico }} &
Total de horas teóricas: {{ memoria_total_horas_teorico }}h
\end{tabularx}
}

% --- Bloco final com memória + atividades ao lado do versículo ---
\noindent
\begin{tabularx}{\linewidth}{@{}X X@{}}
% Coluna da esquerda (memória + atividades)
\begin{minipage}[b]{\linewidth}
    \par
    {\fontsize{10pt}{12pt}\selectfont
\noindent
\textbf{Atividades/Setores:}\\
SM – Saúde Mental\\
MPI – Metodologia de pesquisa I\\
PPP – Pré-parto, parto e pós-parto\\
CCH – Complementação de carga horária (PPP)\\
EAP – Encontro de alinhamento pedagógico\\
}
\end{minipage}
&
% Coluna da direita (versículo)
\begin{minipage}[b]{\linewidth}
    \vspace{0pt} % alinha pela base
    \begin{flushright}
        {\fontsize{10pt}{12pt}\selectfont
\begin{flushright}
    “Assim, fixamos os olhos, não naquilo que se vê, \\
    mas no que não se vê, pois o que se vê é transitório, \\
    mas o que não se vê é eterno.” \\
    \vspace{0.2cm}
    \textbf{2 Coríntios 4:18}
\end{flushright}
}
    \end{flushright}
\end{minipage}
\end{tabularx}

\end{landscape}
\end{document}